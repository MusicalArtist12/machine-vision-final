% Julia Abdel-Monem, 2025

\documentclass[12pt]{article}

\usepackage[left=1in, right=1in, top=1in, bottom=1in]{geometry}
\usepackage{fancyhdr}
\usepackage[export]{adjustbox}
\usepackage[most]{tcolorbox}

\usepackage{setspace}

\renewcommand{\baselinestretch}{1.5}

\newtcolorbox{myframe}[1][]{
  enhanced,
  arc=0pt,
  outer arc=0pt,
  colback=white,
  boxrule=0.8pt,
  #1
}

\newcommand{\diagram}[1]{
    \begin{myframe}
        \includegraphics[width = 6in]{#1}
    \end{myframe}
}

\title{CS455 Final Project}
\author{Julia Abdel-Monem}
\date{March 31, 2025}

\begin{document}
    \maketitle

    \section*{Abstract}


    \section{Introduction}

    \subsection*{Background}

    \subsubsection*{Tailgating}

    According to the NHTSA in 2022, speeding accounted for 29\% of accident related fatalities, and 7.8\% of all fatalities involved at least one distracted driver \cite{NHTSA:Overview-2022}. While there are strategies to avoid crashes with an inattentive or aggressive driver, three studies released by the highway loss data institute (Subaru EyeSight, Kia Drive Wise, and Honda FCW/LDW) demonstrate that crash avoidance features may reduce the number of collision claims made, with young drivers experiencing the most benefit \cite{HLDI:SubaruEyeSight, HLDI:KiaDriveWise, HLDI:HondaFCW}. While the elderly population must still be considered, this trend in age may be explained by overall trends \cite{NHTSA:Overview-2022}. Newer cars are more expensive, so the goal of this project is to develop an app to help lower the financial barrier of entry from a modern car to a relatively recently made smartphone.

    \subsubsection*{Semantic Segmentation}

    Semantic segmentation was chosen since it provides an accurate width, without being overly complex to use. An object detection model such as YOLO would be useful in finding which cars are in front of the camera, but an accurate width requires an accurate edge to read against. instance detection may be useful for this, but it may have more overhead compared to semantic segmentation, where individual cars can be isolated by the width of the lane.

    \subsection*{Neural Network: Design and Training}

    \subsubsection*{The SegFormer}

    The SegFormer is a neural network designed to use vision transformers in semantic segmentation, using multiple transformers using an efficient self attention algorithm and a simple MLP decoder \cite{DBLP:journals/corr/abs-2105-15203}. It builds upon the original patch transformer encoder as described in "An Image is Worth 16x16 Words" \cite{DBLP:journals/corr/abs-2010-11929}. This hierarchical transformer encoder extracts both coarse and fine features. The original SegFormer paper provides five different levels of preconfigured hyperparameters, each one taking more computational power but resulting in a higher IoU.

    \subsubsection*{Intersection over Union}

    \subsubsection*{Dice Loss}

    \subsubsection*{hyperparameters}

    \section{Methods}

    \subsection*{Neural Network}

    \subsubsection*{Implementation Details}

    \subsubsection*{Training}

    \subsection*{Dataset}

    \section{Results}

    \subsection{Binary CrossEntropy, Binary IoU, B0}

    \subsection*{Dice Loss, Binary IoU, no axis set, B0}

    \subsection*{Dice Loss, Binary IoU, Downscaled images, B5}

    \section{Conclusion}

    \bibliographystyle{plain} % We choose the "plain" reference style
    \bibliography{refs} % Entries are in the refs.bib file


\end{document}